
\documentclass[paper=a4,fontsize=11pt]{scrartcl} % KOMA-article class
							
\usepackage[english]{babel}
\usepackage[utf8x]{inputenc}
\usepackage[protrusion=true,expansion=true]{microtype}
\usepackage{amsmath,amsfonts,amsthm}     % Math packages
\usepackage{graphicx}                    % Enable pdflatex
\usepackage[svgnames]{xcolor}            % Colors by their 'svgnames'
\usepackage{geometry}
	\textheight=700px                    % Saving trees ;-)
\usepackage{url}
\usepackage[hidelinks]{hyperref}

\frenchspacing              % Better looking spacings after periods
\pagestyle{empty}           % No pagenumbers/headers/footers

%%% Custom sectioning (sectsty package)
%%% ------------------------------------------------------------
\usepackage{sectsty}

%\sectionfont{%			            % Change font of \section command
%	\usefont{OT1}{phv}{b}{n}%		% bch-b-n: CharterBT-Bold font
%	\sectionrule{0pt}{0pt}{-5pt}{3pt}}

%%% Macros
%%% ------------------------------------------------------------
\newlength{\spacebox}
\settowidth{\spacebox}{8888888888}			% Box to align text
\newcommand{\sepspace}{\vspace*{1em}}		% Vertical space macro

\newcommand{\MyName}[1]{ % Name
		\Huge \usefont{OT1}{phv}{b}{n} \hfill #1
		\par \normalsize \normalfont}
		
\newcommand{\MySlogan}[1]{ % Slogan (optional)
		\large \usefont{OT1}{phv}{m}{n}\hfill \textit{#1}
		\par \normalsize \normalfont}

%\newcommand{\NewPart}[1]{\section*{\uppercase{#1}}}

\newcommand{\PersonalEntry}[2]{
		\noindent\hangindent=2em\hangafter=0 % Indentation
		\parbox{\spacebox}{        % Box to align text
		\textit{#1}}		       % Entry name (birth, address, etc.)
		\hspace{1.5em} #2 \par}    % Entry value

\newcommand{\SkillsEntry}[2]{      % Same as \PersonalEntry
		\noindent\hangindent=2em\hangafter=0 % Indentation
		\parbox{\spacebox}{        % Box to align text
		\textit{#1}}			   % Entry name (birth, address, etc.)
		\hspace{1.5em} #2 \par}    % Entry value	
		
%\newcommand{\EducationEntry}[4]{
%		\noindent \textbf{#1} \hfill      % Study
%		\colorbox{Black}{%
%			\parbox{6em}{%
%			\hfill\color{White}#2}} \par  % Duration
%		\noindent \textit{#3} \par        % School
%		\noindent\hangindent=2em\hangafter=0 \small #4 % Description
%		\normalsize \par}
%
%\newcommand{\WorkEntry}[4]{				  % Same as \EducationEntry
%		\noindent \textbf{#1} \hfill      % Jobname
%		\colorbox{Black}{\color{White}#2} \par  % Duration
%		\noindent \textit{#3} \par              % Company
%		\noindent\hangindent=2em\hangafter=0 \small #4 % Description
%		\normalsize \par}
	
\newcommand{\EducationEntry}[4]{
		\noindent \textbf{#1} \hfill      % Study
		\colorbox{White}{%
			\parbox{6em}{%
		%\parbox{6em}\par
			\hfill\color{Black}#2}} \par  % Duration
		\noindent \textit{#3} \par        % School
		\noindent\hangindent=2em\hangafter=0 \small #4 % Description
		\normalsize \par}

\newcommand{\WorkEntry}[4]{				  % Same as \EducationEntry
		\noindent \textbf{#1} \hfill      % Jobname
		\colorbox{Black}{\color{White}#2} \par  % Duration
		\noindent \textit{#3} \par              % Company
		\noindent\hangindent=2em\hangafter=0 \small #4 % Description
		\normalsize \par}
		
\newcommand{\ProjectEntry}[4]{				  % Same as \EducationEntry
		\noindent \textbf{#1} \hfill      % Jobname
		\colorbox{Black}{\color{White}#2} \par  % Duration
		\noindent \textit{#3} \par              % Company
		\noindent\hangindent=2em\hangafter=0 \small #4 % Description
		\normalsize \par}
		
\newcommand{\ReferenceEntry}[2]{      % Same as \PersonalEntry
		\noindent\hangindent=4em\hangafter=0 % Indentation
		\parbox{\spacebox}{        % Box to align text
		\textit{#1}}			   % Entry name (birth, address, etc.)
		\hspace{1.5em} #2 \par}    % Entry value	

%%% Begin Document
%%% ------------------------------------------------------------
\begin{document}
% you can upload a photo and include it here...
%\begin{wrapfigure}{l}{0.5\textwidth}
%	\vspace*{-2em}
%		\includegraphics[width=0.15\textwidth]{photo}
%\end{wrapfigure}

\textit{Curriculum Vitae}
\MyName{Jared G. Galloway}
\MySlogan{Scientific Programmer | Evolutionary Biologist}
\sepspace

\sepspace

%%% Personal details
%%% ------------------------------------------------------------
\section*{Personal details}{
}
\PersonalEntry{Address}{690 West 11th Eugene, OR}
\PersonalEntry{Phone}{(406) 579-6768}
\PersonalEntry{Mail}{\url{jgallowa@cs.uoregon.edu}}
\PersonalEntry{Website}{\href{https://www.jaredgalloway.org}{www.jaredgalloway.org}}
\PersonalEntry{Github}{\href{https://github.com/jgallowa07}{jgallowa07}}

%%% Education
%%% ------------------------------------------------------------
\section*{Education}{}

\EducationEntry{BSc. Computer and Information Science, Honors}{2014-2018}{University of Oregon}
{3.9 Major GPA. Graduated with departmental honors upon completion of an undergraduate thesis:
\textit{Speeding up the Tortoise}
A Case Study in Optimizing Forward-Moving Evolutionary Simulations (see Research Projects).
My interest in biological mechanisms led me to work with Peter Ralph studying population
genetics through the use of large, forward moving simulations.}
\sepspace

\EducationEntry{Minor in Product Design}{2014-2018}{University of Oregon}
{My background in user-product interaction fuels my interest in building tools (primarily software)
that provide a natural feeling solution to the obstacle at hand. I consider design to be a increasingly
important aspect of software engineering as more of the scientific community 
becomes reliant on technological solutions to diverse problems. 
}
\sepspace

\EducationEntry{Honors Diploma}{2009-2013}{Bozeman High School}
{completed 300+ hours of community service through Montana Conservation Corps.
Advanced Placement courses in Biology, Psychology, and Micro Economics.
}
\sepspace

%%% Work experience
%%% ------------------------------------------------------------
\section*{Work experience}{}

\EducationEntry{Scientific Programmer}{2018-present}{Institute of Ecology and Evolution, University of Oregon, Full-time}
{To further investigate my interests in computational biology, I have been working full time with Andy Kern and Peter Ralph
making pipelines to train neural networks with simulated genomic data. I use my background in data structures and algorithms 
to create efficient methods of storing, processing, and feeding large data sets into deep learning models.
(see Research Projects)}
\newpage

\EducationEntry{Research Assistant}{2017-2018}{Mathematics department, University of Oregon, Full-time}
{As an introduction to population genetics and simulations, I worked with Peter Ralph and William Cresko exploring 
the dynamics of rapid and parallel local adaptation of stickleback fish populations in Alaska. Using large,
forward population genetic simulations, I was tasked with producing and analyzing
large data sets through the use of plotting and summary statistics. 
(see Research Projects)}
\sepspace

\EducationEntry{Discrete Math Grader}{2015-2017}{University of Oregon, Part-time}
{Graded assignments for two series of discrete mathematics class. The material was 
centered around an introduction to proofs, combinatorics, and graph theory.
}

%%% Skills
%%% ------------------------------------------------------------
\section*{Skills}{}

\SkillsEntry{Preferred Workflow}{Unix CLI, Git, Vim, \LaTeX, Slack}
\sepspace

\SkillsEntry{Programming Languages}{Python (fluent)}
\SkillsEntry{}{C/C++ (proficient)}
\SkillsEntry{}{Bash (proficient)}
\SkillsEntry{}{Makefiles (proficient) - I like cmake}
\SkillsEntry{}{Java (proficient)}
\SkillsEntry{}{R (proficient)}
\SkillsEntry{}{Haskell (intermediate)}
\SkillsEntry{}{Javascript (intermediate)}
\sepspace

\SkillsEntry{Mathematics}{Calculus I/II/III}
\SkillsEntry{}{Linear Algebra I/II}
\SkillsEntry{}{Discrete Math}
\SkillsEntry{}{Graph Theory}
\SkillsEntry{}{Statistics}
\sepspace

\SkillsEntry{Evolutionary Simulators}{SLiM, msprime}
\sepspace

\SkillsEntry{Deep Learning Packages}{Keras, Tensor Flow, numpy}
\sepspace

%%% Skills
%%% ------------------------------------------------------------
\section*{Research Projects}{}
\EducationEntry{Deep leaning to infer population genetic parameters}{2018-present}{In progress}
{I am currently working with Drs.\ Andy Kern and Peter Ralph, using deep learning to infer population genetic parameters 
(starting with recombination rates, for now) given the
genotype matrix of a set of samples from a simulated population.  Currently, vanilla architectures
such as an LSTM are competitive to industry standards such as \textbf{LD Hat} when trained and tested 
on basic coalescent simulations. While impressive, we are not only interested in what deep learning \textit{can} do, 
but also what limitations exist in terms of dealing with more complex demographic data.
For this project, my role is building the pipelines which efficiently generate the data for neural networks to train on.
This involves writing, testing and profiling code that can be repurposed
 as well as exploring data structures and algorithms best suited for the hardware available. 
}
\sepspace

\EducationEntry{Tree-sequence recording in SLiM opens new horizons for forward-time simulation of whole genomes}{2017-2018}{Accepted | Molecular Ecology Resources}
{In this project I worked with Drs.\ Peter Ralph, Ben Haller, Jerome Kelleher, and Philipp Messer to 
implement and describe the applications of genealogical tree sequence recording (\textit{TreeSeq}) in \textbf{SLiM 3.0}.
This includes the types of simulations which could benefit from the ability to avoid tracking neutral mutations.
Many examples in the paper also describe how to read in the tree sequence produced into python and use the
\textbf{msprime} API to extract a multitude of information from the object. On this project, I was involved in  
Conceptualization, Methodology, Software, Writing, Review \& Editing. 
}
\url{https://onlinelibrary.wiley.com/doi/abs/10.1111/1755-0998.12968?af=R}
\sepspace


\EducationEntry{Undergraduate Thesis, \textit{Speeding up the tortoise}}{2017-2018}{Accepted by department}
{For my undergraduate thesis, I implemented the first stages of genealogical tree sequence recording (\textit{TreeSeq}) for 
a forward moving evolutionary simulator, \textbf{SLiM} $3.0$.
Using C/C++, our lab integrated the data structures and algorithms used in the backend of \textbf{msprime} for the 
\textit{treeSequence} object with the core code of SLiM. By giving simulations the ability to avoid tracking neutral mutations,
\textit{TreeSeq} resulted in one-to-two orders of magnitude speedup in certain simulations. 
In the thesis, I describe the concepts behind genealogical tree sequence recording, the algorithms 
and data structured needed to implement it in a forward moving simulation, and optimizations made 
to improve it. The paper does not describe testing and compilation methods used to bring the software together.
My advisors for this thesis were Boyana Norris and Peter Ralph
}
\url{https://www.cs.uoregon.edu/Reports/UG-201806-Galloway.pdf}
\sepspace


\EducationEntry{A few stickleback suffice to transport adaptive alleles to new lakes}{2017-2018}{Final stages of editing for submission to Genetics}
{In this project, I am working with Drs.\ William Cresko and Peter Ralph to explore rapid adaptation of Alaskan Stickleback populations 
through the use of large, forward moving population genetics simulations. 
Threespine Stickleback fish provide a striking example of local adaptation in the context of recurrent gene flow. 
This Northern hemisphere-wide metapopulation includes both marine populations and a large number of smaller freshwater populations that have repeatedly adapted to freshwater conditions, often by using standing genetic variation. 
Here we determine the levels of gene flow can best match the observed patterns of allele sharing among habitats in stickleback, 
and to provide a framework for better understanding of the dynamics of gene flow and local adaptation for the maintenance and reuse of standing genetic variation.
}
\url{https://github.com/jgallowa07/SticklebackPaper/blob/master/Stickleback_Paper.pdf}
\sepspace

\EducationEntry{Squirrel Suiter}{2016}{Google Play app store}
{Squirrel Suiter is an ``endless flyer'' game where you play as a jetpack-equipped flying squirrel,
 avoiding obstacles and munching acorns to gain the highest score. My contributions to this project include
 Project Manager, Physics, Linear Algebra,  and Level Design  }
\url{https://play.google.com/store/apps/details?id=com.Nighthawks.SquirrelSuiter}
\sepspace
\newpage

\EducationEntry{Image manipulator}{2016}{github}
{This is an object oriented, source-sink, approach to image augmentation written purely in C++ with only the vector library.
It takes in pnm images, applies a desired set of filters to it, and returns the augmented image. 
}
\url{https://jaredgalloway.org/Image-Manipulator}
\sepspace


%%% References
%%% ------------------------------------------------------------
\section*{References}

\subsubsection*{| Dr. Peter L. Ralph} 
\begin{center}
Research Advisor (University of Oregon) 
plr@uoregon.edu
\end{center}

\subsubsection*{| Dr. William R. Cresko}  
\begin{center}
Research Collaborator, (University of Oregon).
wcresko@uoregon.edu
\end{center}

\subsubsection*{| Dr. Andrew D. Kern}
\begin{center}
Research Advisor (University of Oregon),
adkern@uoregon.edu
\end{center}

\subsubsection*{| Dr. Benjamin C Haller} 
\begin{center}
Research Collaborator (Cornell University),
bhaller@mac.com
\end{center}

\subsubsection*{| Eric Merchant} 
\begin{center}
Instructor and Employer (University of Oregon), 
ericm@uoregon.edu
\end{center}


\end{document}










