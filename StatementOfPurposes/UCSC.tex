\documentclass[11pt]{amsart}
\usepackage[margin=1.3in]{geometry} 
\geometry{letterpaper}                   % ... or a4paper or a5paper or ... 
%\geometry{landscape}                % Activate for for rotated page geometry
%\usepackage[parfill]{parskip}    % Activate to begin paragraphs with an empty line rather than an indent
\usepackage{graphicx}
\usepackage{amssymb}
\usepackage{epstopdf}
\usepackage{url}
\usepackage{fancyhdr}
 
\pagestyle{fancy}
\fancyhf{}
\lhead{Page \thepage}
%\rhead{Jared G. Galloway}
\rfoot{Jared G. Galloway}

\DeclareGraphicsRule{.tif}{png}{.png}{`convert #1 `dirname #1`/`basename #1 .tif`.png}

\title{UC Santa Cruz Biomolecular Engineering and Bioinformatics track Ph.D. Statement of Purpose}
\author{Jared G. Galloway}
%\date{}                                           % Activate to display a given date or no date
\newlength{\spacebox}
\settowidth{\spacebox}{8888888888}			% Box to align text
\newcommand{\sepspace}{\vspace*{1em}}		% Vertical space macro

\begin{document}
\maketitle
%\section{}
%\subsection{}


Biomolecular engineering and bioinformatics research has opened new horizons for human health and understanding of the forces driving evolution.
With recent advances in technology, quantitative biology research, and access to large datasets, 
the field has become more equipped to solve complex problems that will play a large role in current and future generations to come.
I am applying to graduate school because I believe it is a necessary challenge to nourish my biological skill set and interest in quantitative problem-solving. 
This training will present an opportunity for me to leverage my computational background and to help find creative solutions to problems facing the field at an exciting time for biology.
Moving forward, I would like to help progress the field and participate in the discussions that define our understanding of complex systems.
My professional goals are to build tools (software) that biologists can use to help solve a wide variety of problems in genetics research.
%Genomics research has opened new horizons for human health and the understanding of our origins.
%With recent advances in technology, genetics research, and access to large datasets, the field has become more equipped to solve complex problems for current and future generations.
%(1) I'm applying to graduate school because I believe it is a necessary challenge to nourish my biological skill set and interest in quantitative problem-solving. 
%This training will present an opportunity for me to leverage my computational background and help find creative solutions to problems facing the field at an exciting time for biology.
%(3) Moving forward, I would like to help progress the field and participate in the discussions that define our understanding of complex systems.
%(2) My professional goals are to build tools (software) that biologists can use to help solve a wide variety of problems in genetics research.
\sepspace

%(2) A large portion of the results I have produced thus far, have relied heavily on the software which has been released to the public.
%My latest research project has involved using deep learning to infer recombination rates. 
%For this and many other related projects, my lab is dependent on a range of tools: 
%Tensor Flow for efficient machine learning, 
%SLiM and msprime for simulating large and complex datasets,
%and a broad spectrum of software for analyzing results.  
%Using these tools, we are currently outcompeting with industry standards such as LD Hat.
%Software engineering has proven to be an extremely effective method at abstracting complex scientific concepts into convenient building blocks such that researchers can tailor them to their specific needs.
%Working on the implementation of \textit{Tree Sequence Recording} in SLiM (see CV for details), has sparked my interest in
%providing, maintaining, and improving the tools that other researchers may use to make progress in the field.
%\sepspace

Over the course of the past two years I have completed two major projects and am currently working as a full-time scientific programmer in the Institute of Ecology and Evolution at the University of Oregon.
Advised by Dr. Peter Ralph and Dr. William Cresko, the first project I took the lead on was studying local polygenic adaptation of stickleback fish populations in Alaska. 
This northern hemisphere-wide metapopulation includes both marine populations and a large number of smaller freshwater populations that have repeatedly adapted to freshwater conditions, often by using standing genetic variation. 
For this project we wanted to know what range of introgression between marine and freshwater populations was required to maintain the transportation of freshwater alleles.
We were also interested in the genetic signals we can expect to see in real data and how gene flow impacts the those signals. 
Using SLiM, I wrote code to run large evolutionary simulations which emulated the geography and evolutionary history of stickleback populations in Alaska.
We then varied the gene flow by changing migration rates to observe the impact of selection on standing genetic variation. 
From this we found that rapid, repeated adaptation using alleles maintained at low frequency by migration-selection balance occurs over a realistic range of intermediate rates of gene flow.
We outlined the rates of gene flow which allowed us to see causal loci from $F_{st}$ scans across the genome as well as the rates at which migration load prevents adaptation.
Lastly, we traced back to the origin of all alleles which came to high frequency in the introduced populations after adaptation had occurred, and found the majority were pre-existing in the first generation as opposed to being carried in by subsequent migration.
As a first author, the manuscript for this paper is in the final stages of editing and will be posted to the BioArxiv soon, then submitted to Genetics. The most recent draft can be found at
\url{https://github.com/jgallowa07/SticklebackPaper/blob/master/Stickleback_Paper.pdf}.
\sepspace

Working with Dr. Ben Haller and Dr. Peter Ralph on the implementation and profiling  of \textit{Tree Sequence Recording} in SLiM 3.0, was a large part of my undergraduate thesis.
Genealogical tree sequence recording is a strategy for efficiently recording the genealogical history from forward-moving simulations. 
This history is represented by the forest of trees relating all sampled individuals to each other over every genomic interval. 
\textit{TreeSeq} uses a collection of tabular data structures to encode this history which was introduced for use in the coalescent simulator msprime. 
Using TreeSeq, simulations in SLiM can avoid the cost of tracking and propagating neutral mutations as a by-product of obtaining the origins of all sampled genotypes.
%For this project our goal was to improve on the simulation software by tracking the genealogical history of all sampled (commonly extant) individuals.
%By we may avoid the computationally expensive task of tracking neutral mutations
%This strategy is possible made possible by tracking the outcome of every meiosis throughout the simulation, and pruning the genealogical trees to include only the history of sampled individuals. 
%By definition neutral mutations do not effect the outcome of the population process, this means we can simply lay the neutral mutations over the gene trees after the simulation has finished, and yield statistically identical results.
For this project we utilized a variety of software engineering tools including C/C++, cmake, xcode and an agile workflow.
After successfully implementing \textit{TreeSeq} with rigorous testing, we found simulations where individuals had realistic size genomes experienced a speedup of over 2 orders of magnitude. 
The paper describing the applications of this strategy, titled \textit{Tree-sequence recording in SLiM opens new horizons for forward-time simulation of whole genomes}, was published by Molecular Ecology Resources and can be found online at 
\url{https://onlinelibrary.wiley.com/doi/abs/10.1111/1755-0998.12968?af=R}.
\sepspace
%This majority of coding done for this project was done in c++, and gave me experience with compilation methods using cmake and coding practices on a widely distributed piece of software. 

My current project working with Dr. Andrew Kern involves using deep learning to infer population genetics parameters from sampled data. 
For this project, we want to know if the recent advances in deep learning architecture can learn complex patters in genotype matrices resulting in accurate predictions of recombination and mutation rates. 
Using Tensor Flow and other data analysis packages for python, I have set up a pipeline used for: 
simulating and storing large datasets efficiently, 
concurrently prepping data batches while training on the previously generated batch, 
and testing the performance of trained neural networks.
Using this pipeline, we have found architectures and data prepping heuristics which have resulted in predictions of recombination rates that consistently outperform industry standards such as LD Hat. 
%Previous work by has shown that sorting individuals in the matrix by similarity then applying using temporal convolutions (1D CNN) has shown promise in predicting 
\sepspace

%**
%An overview of what you wanted to learn (state your hypothesis).
%--whether population genetics parameters can be inferred from population genetics sampling (in this case, the genotype matrix)

%A summary of your research methods and results.
%--Deep learning, 

%An explanation of what you have learned.
%--Organized, modular code

%Additionally, the Committee would like to know WHY you want to pursue a Ph.D., 
%what excites you about science and what research areas you would like to focus on as a graduate student.
%**
The projects described above have exposed me to a wide variety of research environment practices including organization of experiments, scientific writing, seeking and approaching others when I need help, and effectively communicating my work. 
I have gained a familiarity with many concepts in genomics, population genetics, and applied computer science skills such as simulation and machine learning.
%(1) My undergraduate degree in computer science has provided a background in algorithms and data structures best suited for particular styles of problems, as well as the programming skills needed to implement them. 
%The research I have participated in has also brought to light the expanding variety of problems in genomics for which a computational approach could be utilized.
%A large number of problems in genomics today are presented in the form of analyzing massive genomic datasets that have recently become much more accessible with efficient sequencing. 
%As the data available grows larger, we stand to gain an immense amount of insight using novel computing methods.
%The current advances produced by the biology experts, along with the flood of data we anticipate, has shaped an exciting future for genetics that I am eager to be a part of. 
Through graduate training in biomolecular engineering and bioinformatics, I aspire to be truly interdisciplinary. 
Lying at the intersection of computer science and biology, I strive to effectively communicate with experts on both sides so as to bridge the fields.  
%As discussions surrounding things such as gene therapy, precision medicine, and genomic security become closer in proximity to our reality, I hope that my research will contribute to the progression of quantitative biology.
The current advances produced by biology experts along with the flood of data we anticipate has shaped an exciting future for genetics that I am eager to be a part of. 
The wide range of study that PBSE offers ranging from paleogenomics to precision medicine, would allow me to further explore where my skill set intersects with my interests.
%I believe my background in computation and design, in conjunction with graduate training, will make me fit for approaching many of the challenges that face research in genetics.
\sepspace

If shortlisted as a candidate, I would like the opportunity to speak with David Haussler, Benedict Paten, and Beth Shapiro


\end{document}  