\documentclass[12pt]{amsart}
\usepackage[margin=0.7in]{geometry}                % See geometry.pdf to learn the layout options. There are lots.
\geometry{letterpaper}                   % ... or a4paper or a5paper or ... 
%\geometry{landscape}                % Activate for for rotated page geometry
%\usepackage[parfill]{parskip}    % Activate to begin paragraphs with an empty line rather than an indent
\usepackage{graphicx}
\usepackage{amssymb}
\usepackage{epstopdf}
\DeclareGraphicsRule{.tif}{png}{.png}{`convert #1 `dirname #1`/`basename #1 .tif`.png}

\title{Statement of Purpose: University of Oregon Biology Ph.D.}
\author{Jared G. Galloway}
%\date{}                                           % Activate to display a given date or no date

\newlength{\spacebox}
\settowidth{\spacebox}{8888888888}			% Box to align text
\newcommand{\sepspace}{\vspace*{1em}}		% Vertical space macro

\begin{document} %one page
\maketitle

%\section{}
%\subsection{}
% (1) In approximately one page, explain why you are applying to graduate school, 
% (2) what you would like to accomplish while in a graduate program, 
% (3) and describe your professional goals.
\sepspace

%I find quantitative biology to be a creative field
%I consider myself to be a creative thinker and find quantitative biology to be a creative field. 
Quantitative biology research has opened new horizons for human health and understanding of the forces driving evolution.
With recent advances in technology, genetics research, and access to large datasets, %we are presented with a limitless supply of . %and to face unknown adversity
the field has become more equipped to solve complex problems that will play a large role in current and future generations to come.
%the field of genetics is becoming more equipped to solve complex problems for current and future generations.
I am applying to graduate school because I believe it is a necessary challenge to nourish my biological skill set and interest in quantitative problem-solving. 
This training will present an opportunity for me to leverage my computational background and to help find creative solutions to problems facing the field at an exciting time for biology.
%Currently, my interests lie in the intersection of genetics, machine learning, and software engineering.
%I hope to impact the field of computational biology in a way that allows others to build off of my work. 
%I hope to interact with a network of colleagues, with variety of background knowledge, who will offer new perspectives and force me to better communicate my own.
Moving forward, I would like to help progress the field and participate in the discussions that define our understanding of complex systems.
My professional goals are to build tools (software) that biologists can use to help solve a wide variety of problems in genetics research.

%impact the field of computational biology in a way that allows others to build off of my work. 

%Concretely, I would like to build tools (software) that will be used by biologists to help solve a wide variety of problems in quantitative genomics. 
\sepspace

%To effectively learn the skills that allow me to navigate complex problems while leading projects, I believe graduate training is a necessity. 
%It is obvious that in the field of quantitative and computational biology I will face problems which require a broad range of skills including but not restricted to
%communication, quantitative reasoning, and creativity, all of which I hope my Ph.D. program will effectively train me to excel in. 
My undergraduate degree in Computer and Information Science has provided a background in algorithms and data structures best suited for particular styles of problems, as well as the programming skills needed to implement them. 
%Additionally, I have to come to greatly enjoy finding solutions as well as producing tangible results. 
%My background in quantitative problems has provided me with the skills to approach many flavors of quantitative problems
%My research in population genetics and evolutionary simulations has shown me the wide variety of problems in genomics for which a computational approach is best suited for.
The research I have participated in has brought to light the wide variety of problems in genomics for which a computational approach is necessary.
These problems include things like inferring genealogical ancestry given a set of sample haplotypes, or training neural networks to make predictions on causal loci.
%I hope that graduate school will push me to excel in all aspects of quantitative and computational biology. 
%A large number of these problems are presented in the form of analyzing massive genomic datasets that have recently become much more accessible with efficient sequencing. 
As sequencing becomes more efficient, and the number of samples grows larger, we stand to gain an immense amount of insight using modern computing methods.
%However, the limitation of computational power and time brings about a need for software engineering that incorporates code which is scaleable and modular in design.
%This is where I believe my background, paired with a formal graduate training in biology, is a great fit to approach these problems.
The current advances produced by biology experts, along with the flood of data we anticipate, has shaped an exciting future for genetics that I am eager to be a part of. 
%My hope is that with formal training in biology, I'll be in a fit to approach a new era of data processign 
%During graduate training, I am confident that I can contribute to the positive, forward progression of quantitative biology,.
%I'm applying to this \textit{biology} Ph.D. program because  ... %I personally have found biological mechanisms, specifically genetics, to be the most interesting and closest in proximity as humans
%LAST?With a background in computation and design, I believe graduate training will present me with the opportunity to contribute my knowledge to a wide variety of obstacles facing computational biology
\sepspace


A large portion of the results I have produced thus far, have relied heavily on the software which has been released to the public.
My latest research project has involved using deep learning to infer recombination rates. 
For this and many other related projects, my lab is dependent on a range of tools: 
Tensor Flow for efficient machine learning, 
SLiM and msprime for simulating large and complex datasets,
and a broad spectrum of software for analyzing results.  
Using these tools, our trained neural nets are currently outcompeting with industry standards such as LD Hat.
%Publicly distributed software makes it possible to do high level experiments with a relatively small amount of code.
%When using machine learning to infer population genetics parameters or simulation to create massive datasets . 
%From deep learning with Tensor Flow, to running coalescent simulations with Msprime, 
Software engineering has proven to be an extremely effective method at abstracting complex scientific concepts into convenient building blocks such that researchers can tailor them to their specific needs.
%Using Tensor Flow for deep learning, I have constructed pipelines which will train a multitude of neural networks on large datasets without needing to waste large amounts of time implementing activation functions or Backpropagation.
%The impact of software on science, in conjunction with my background in design, is why building software especially sparks my interest moving forward
%To effectively and efficiently produce results, we rely heavily on the tools which have been provided.
Working on the implementation of \textit{Tree Sequence Recording} in SLiM (see CV for details), has sparked my interest in
providing, maintaining, and improving the tools that other researchers may use to make progress in the field.
%Working with the variety of people both in and outside of my current lab environment has exposed the 
%importance of building a network of colleagues who share common interests with different backgrounds.  
%During graduate school, I would like to meet people that can provide a different perspective and/or critique my work.
\sepspace

Through graduate training in biology, I aspire to be truly interdisciplinary. 
Lying at the intersection of computer science and biology, I strive to effectively communicate with experts on both sides, so as to bridge the fields.  
As discussions surrounding things such as gene therapy, precision medicine, and genomic security become closer in proximity to our reality, I hope that my research will contribute to the progression of quantitative biology.
The Institute of Ecology and Evolution at the University of Oregon has proven to maintain close ties with other departments such that we may see the full benefit of working with other fields.
I believe working in labs such as the Ralph and Kern lab will present an opportunity for me launch my career.
%I believe my background in computation and design, in conjunction with graduate training, will make me fit for approaching many of the challenges that face research in biology.
%In conjunction with my background in design, this is why building software especially sparks my interest. 
%One aspect of science that most intrigues me is how it builds on itself. 
%I enjoy how the nature of academia is to provide other researchers with methods and results which could help with similar and quite different problems. 

%In conjunction with my background in design, this is why building software especially sparks my interest. 
%To effectively and efficiently produce results we rely heavily on the tools which have been provided.
%Working on implementing \textit{Tree Sequence Recording} in SLiM (see CV for details), has brought to light how exciting it is to 
%provide, maintain, and improve the tools that others use, and the impact it makes on the field. 





\end{document}  